\documentclass[12pt,en,a4paper]{article}
\usepackage{vntex}
\usepackage{enumerate}
\usepackage{array}
\usepackage{enumitem}
\usepackage{hyperref}

\begin{document}
	\begin{table}[]
		\centering
		\begin{tabular}{|p{0.1\textwidth}|p{0.5\textwidth}|p{0.2\textwidth}|p{0.2\textwidth}|}
			\hline
			\multicolumn{4}{|c|}{GROUP --- MEMBER LIST} \\
			\hline
			No. & Name & ID & Role \\
			\hline
			1 & Nguyễn Hoàng & 1952255 & Blank \\
			\hline
			2 & Blank & Blank & Blank \\
			\hline
		\end{tabular}
	\end{table}
\pdfbookmark[0]{Problem 1}{prob1}
	\section*{Problem 1:}
	Let p and q be the propositions "The election is decided" and "The votes have been counted" respectively.
	Express each of these compound propositions as English sentences.
	
	\begin{enumerate}
		\item $\neg$p: The election is not decided
		\item p $\vee$ q: The election is decided or the votes have been counted
		\item $\neg$p $\wedge$ q: The election is not decided and the votes have been counted
		\item q $\rightarrow$ p: The votes have been counted implies the election is decided
		\item $\neg$q $\rightarrow$ $\neg$p: The votes have not been counted implies the election is not decided
		\item $\neg$p $\rightarrow$ $\neg$q: The election is not decided implies the votes have not been counted
		\item p $\leftrightarrow$ q: The election is decided if and only if the votes have been counted
		\item $\neg$q $\vee$ ($\neg$p $\wedge$ q): The votes have not been counted or the election is not decided and the votes have been counted
	\end{enumerate}
\newpage
\pdfbookmark[0]{Problem 2}{prob2}
	\section*{Problem 2:}
	Construct a truth table for the compound propositions ((p $\rightarrow$ q) $\rightarrow$ r) $\rightarrow$ s
	
	\begin{tabular}{|c|c|c|c|c|c|c|}
		\hline
		p & q & r & s & p $\rightarrow$ q & (p $\rightarrow$ q) $\rightarrow$ r & ((p $\rightarrow$ q) $\rightarrow$ r) $\rightarrow$ s \\
		\hline
		T & T & T & T & T & T & T\\
		\hline
		T & T & T & F & T & T & F\\
		\hline
		T & T & F & T & T & F & T\\
		\hline
		T & T & F & F & T & F & T\\
		\hline
		T & F & T & T & F & T & T\\
		\hline
		T & F & T & F & F & T & F\\
		\hline
		T & F & F & T & F & T & T\\
		\hline
		T & F & F & F & F & T & T\\
		\hline
		F & T & T & T & T & T & T\\
		\hline
		F & T & T & F & T & T & F\\
		\hline
		F & T & F & T & T & F & T\\
		\hline
		F & T & F & F & T & F & T\\
		\hline
		F & F & T & T & T & T & T\\
		\hline
		F & F & T & F & T & T & F\\
		\hline
		F & F & F & T & T & F & T\\
		\hline
		F & F & F & F & T & F & T\\
		\hline
	\end{tabular}
\newpage
\pdfbookmark[0]{Problem 3}{prob3}
	\section*{Problem 3:}
	Show that the statement (p $\vee$ q) $\wedge$ ($\neg$p $\wedge$ $\neg$q) is a contradiction
	
	(p $\vee$ q) $\wedge$ ($\neg$p $\wedge$ $\neg$q)\\
	= p $\wedge$ $\neg$p $\wedge$ $\neg$q $\vee$ q $\wedge$ $\neg$p $\wedge$ $\neg$q\\
	= \textbf{F} $\vee$ \textbf{F}\\
	= \textbf{F}
\newpage
\pdfbookmark[0]{Problem 4}{prob4}
	\section*{Problem 4:}
	On the island of Flopi, there are three types of people: Knights, Knaves, and Floppers.
	All inhabitants know which type the others are, but they are otherwise indistinguishable. Knights always
	tell the truth. Knaves always lie. Floppers always choose to lie or tell the truth by doing the opposite of
	the previous speaker (i.e. if someone just spoke a lie, the 
Flopper will tell the truth; if someone just spoke a
	truth, the 
Flopper will lie). While on your vacation, you come across three inhabitants, A, B, and C. They
	say the following, in order:\\
	A says, "We are all knights."\\
	B says, "C is a knight."\\
	C says, "A is a knave."\\
	A says, "C lied."\\\\
	If A is a knight, statement 1 and 3 contradict. Therefore A is not a knight.	
	If A is a knave:
	\begin{itemize}
		\item If B is a knight, C is a knight.
		\item If B is a flopper, C is a knight.
	\end{itemize}
	Since statement 1 is false, the order can be flopper - knight - knight or flopper - flopper - knight.
\newpage
\pdfbookmark[0]{Problem 5}{prob5}
	\section*{Problem 5:}
	Five friends have access to a chat room. Is it possible to determine who is chatting if
	the following information is known? Either Kevin or Heather, or both, are chatting. Either Randy or Vijay,
	but not both, are chatting. If Abby is chatting, so is Randy. Vijay and Kevin are either both chatting or
	neither is. If Heather is chatting, then so are Abby and Kevin.\\\\
	Let Kevin, Heather, Randy, Vijay, Abby chatting be K, H, R, V, A respectively.\\
	Let S\textsubscript{1} be "Either Kevin or Heather or both are chatting."\\
	Let S\textsubscript{2} be Either Randy or Vijay, but not both, are chatting."\\
	Let S\textsubscript{3} be "If Abby is chatting, so is Randy."\\
	Let S\textsubscript{4} be "Vijay and Kevin are either both chatting or neither."\\
	Let S\textsubscript{5} be "If Heather is chatting, then so are Abby and Kevin."
	
	We have:\\
	S\textsubscript{1} = K $\vee$ H\\
	S\textsubscript{2} = R $\oplus$ V\\
	S\textsubscript{3} = A $\rightarrow$ R\\
	S\textsubscript{4} = V $\leftrightarrow$ K\\
	S\textsubscript{5} = H $\rightarrow$ (A $\wedge$ K)\\\\
	Conclusion C = S\textsubscript{1} $\wedge$ S\textsubscript{2} $\wedge$ S\textsubscript{3} $\wedge$ S\textsubscript{4} $\wedge$ S\textsubscript{5}
	
	\begin{tabular}{|c|c|c|c|c|c|c|c|c|c|c|}
		\hline
		K & H & V & R & A & S\textsubscript{1} & S\textsubscript{2} & S\textsubscript{3} & 
		S\textsubscript{4} & 
		S\textsubscript{5} & 
		C\\
		\hline
		T & T & T & T & T & T & F & T & T & T & F\\
		\hline
		T & T & T & T & F & T & F & T & T & F & F\\
		\hline
		T & T & T & F & T & T & T & F & T & T & F\\
		\hline
		T & T & T & F & F & T & T & T & T & F & F\\
		\hline
		T & T & F & T & T & T & T & T & F & T & F\\
		\hline
		T & T & F & T & F & T & T & T & F & F & F\\
		\hline
		T & T & F & F & T & T & F & F & F & T & F\\
		\hline
		T & T & F & F & F & T & F & T & F & F & F\\
		\hline
		T & F & T & T & T & T & F & T & T & T & F\\
		\hline
		T & F & T & T & F & T & F & T & T & T & F\\
		\hline
		T & F & T & F & T & T & T & F & T & T & F\\
		\hline
		T & F & T & F & F & T & T & T & T & T & T\\
		\hline
		T & F & F & T & T & T & T & T & F & T & F\\
		\hline
		T & F & F & T & F & T & T & T & F & T & F\\
		\hline
		T & F & F & F & T & T & F & F & F & T & F\\
		\hline
		T & F & F & F & F & T & F & T & F & T & F\\
		\hline
		F & T & T & T & T & T & F & T & F & F & F\\
		\hline
		F & T & T & T & F & T & F & T & F & F & F\\
		\hline
		F & T & T & F & T & T & T & F & F & F & F\\
		\hline
		F & T & T & F & F & T & T & T & F & F & F\\
		\hline
		F & T & F & T & T & T & T & T & T & F & F\\
		\hline
		F & T & F & T & F & T & T & T & T & F & F\\
		\hline
		F & T & F & F & T & T & F & F & T & F & F\\
		\hline
		F & T & F & F & F & T & F & T & T & F & F\\
		\hline
		F & F & T & T & T & F & F & T & F & T & F\\
		\hline
		F & F & T & T & F & F & F & T & F & T & F\\
		\hline
		F & F & T & F & T & F & T & F & F & T & F\\
		\hline
		F & F & T & F & F & F & T & T & F & T & F\\
		\hline
		F & F & F & T & T & F & T & T & T & T & F\\
		\hline
		F & F & F & T & F & F & T & T & T & T & F\\
		\hline
		F & F & F & F & T & F & F & F & T & T & F\\
		\hline
		F & F & F & F & F & F & F & T & T & T & F\\
		\hline
	\end{tabular}\\
	Therefore Kevin and Vijay are chatting.
\newpage
\pdfbookmark[0]{Problem 6}{prob6}
	\section*{Problem 6:}
	Show that (p $\rightarrow$ q) $\rightarrow$ [(q $\rightarrow$ r) $\rightarrow$ (p $\rightarrow$ r)] is a tautology using a truth table.\\
	\begin{tabular}{|c|c|c|c|c|c|c|c|}
		\hline
		p & q & r & q $\rightarrow$ r & p $\rightarrow$ r & p $\rightarrow$ q & (q $\rightarrow$ r) $\rightarrow$ (p $\rightarrow$ r) & (p $\rightarrow$ q) $\rightarrow$ [(q $\rightarrow$ r) $\rightarrow$ (p $\rightarrow$ r)]\\
		\hline
		T & T & T & T & T & T & T & T\\
		\hline
		T & T & F & F & F & T & T & T\\
		\hline
		T & F & T & T & T & F & T & T\\
		\hline
		T & F & F & T & F & F & F & T\\
		\hline
		F & T & T & T & T & T & T & T\\
		\hline
		F & T & F & F & T & T & T & T\\
		\hline
		F & F & T & T & T & T & T & T\\
		\hline
		F & F & F & T & T & T & T & T\\
		\hline
	\end{tabular}
\newpage
\pdfbookmark[0]{Problem 7}{prob7}
	\section*{Problem 7:}
	Prove that (p $\rightarrow$ q) $\wedge $(p $\rightarrow$ r) and p $\rightarrow$ (q $\wedge$ r) are logically equivalent.
	
	(p $\rightarrow$ q) $\wedge$ (p $\rightarrow$ r)\\
	= ($\neg$p $\vee$ q) $\wedge$ ($\neg$p $\vee$ r)\\
	= $\neg$p $\vee$ $\neg$p $\wedge$ q $\vee$ $\neg$p $\wedge$ r $\vee$ q $\wedge$ r\\
	= $\neg$p $\vee$ q $\wedge$ r
	
	p $\rightarrow$ (q $\wedge$ r)\\
	= $\neg$p $\vee$ (q $\wedge$ r)\\
\newpage
\pdfbookmark[0]{Problem 8}{prob8}
	\section*{Problem 8:}
	Prove that $\neg$[[[[(p $\wedge$ q) $\wedge$ r] $\vee$ [(p $\wedge$ r) $\wedge$ $\neg$r]] $\vee$ $\neg$q] $\rightarrow$ s] and [(p $\wedge$ r) $\vee$ $\neg$q] $\wedge$ $\neg$s are logically equivalent. Then check it using a truth table.
	
	$\neg$[[[[(p $\wedge$ q) $\wedge$ r] $\vee$ [(p $\wedge$ r) $\wedge$ $\neg$r]] $\vee$ $\neg$q] $\rightarrow$ s]\\
	= $\neg$[$\neg$[[p $\wedge$ q $\wedge$ r $\vee$ p $\wedge$ r $\wedge$ $\neg$r] $\vee$ $\neg$q] $\vee$ s]\\
	= $\neg$[$\neg$[p $\wedge$ q $\wedge$ r $\vee$ $\neg$q] $\vee$ s]\\
	= [p $\wedge$ q $\wedge$ r $\vee$ $\neg$q] $\wedge$ $\neg$s\\
	= [p $\wedge$ q $\wedge$ r $\vee$ $\neg$q $\vee$ p $\wedge$ $\neg$q $\wedge$ r] $\wedge$ $\neg$s\\
	= [p $\wedge$ r $\vee$ $\neg$q] $\wedge$ $\neg$s\\	
	\begin{tabular}{|c|c|c|c|c|>{\bfseries}c|c|>{\bfseries}c|}
		\hline
		p & q & r & s & p $\wedge$ q $\wedge$ r $\vee$ $\neg$q & p $\wedge$ q $\wedge$ r $\vee$ $\neg$q $\rightarrow$ s & p $\wedge$ r $\vee$ $\neg$q & [p $\wedge$ r $\vee$ $\neg$q] $\wedge$ $\neg$s\\
		\hline
		T & T & T & T & T & T & T & T\\
		\hline
		T & T & T & F & T & F & T & F\\
		\hline
		T & T & F & T & F & T & F & T\\
		\hline
		T & T & F & F & F & T & F & T\\
		\hline
		T & F & T & T & T & T & T & T\\
		\hline
		T & F & T & F & T & F & T & F\\
		\hline
		T & F & F & T & T & T & T & T\\
		\hline
		T & F & F & F & T & F & T & F\\
		\hline
		F & T & T & T & F & T & F & T\\
		\hline
		F & T & T & F & F & T & F & T\\
		\hline
		F & T & F & T & F & T & F & T\\
		\hline
		F & T & F & F & F & T & F & T\\
		\hline
		F & F & T & T & T & T & T & T\\
		\hline
		F & F & T & F & T & F & T & F\\
		\hline
		F & F & F & T & T & T & T & T\\
		\hline
		F & F & F & F & T & F & T & F\\
		\hline
	\end{tabular}
\newpage
\pdfbookmark[0]{Rosen's Book}{rosen}
	\section*{Rosen's book}
	\subsection*{1.1 42:}
	What is the value of x after each of these statements is encountered in a computer program, if x = 1 before the
	statement is reached?\\
	\begin{enumerate}[label=(\alph*)]
		\item \textbf{if} $x+2=3$ \textbf{then} $x:=x+1$\\
		$\Rightarrow x=2$
		\item \textbf{if} $(x+1 =3)OR(2x+2=3)$ \textbf{then} $x:=x+1$\\
		$\Rightarrow x=1$
		\item \textbf{if} $(2x+3=5)AND(3x+4=7)$ \textbf{then} $x:=x+1$\\
		$\Rightarrow x=2$
		\item \textbf{if} $(x+1=2)XOR(x+2=3)$ \textbf{then} $x:=x+1$\\
		$\Rightarrow x=1$
		\item \textbf{if} $x<2$ \textbf{then} $x:=x+1$\\
		$\Rightarrow x=2$
	\end{enumerate}
	\subsection*{1.2 6:}
	You can upgrade your operating system only if you have a 32-bit processor running at 1 GHz or faster, at least 1 GB RAM, and 16 GB free hard disk space, or a 64-bit processor running at 2 GHz or faster, at least 2 GB RAM, and at least 32 GB free hard disk space. Express your answer in terms of \textit{u}: "You can upgrade your operating system," \textit{b}\textsubscript{32}:"You have a 32-bit processor," \textit{b}\textsubscript{64}:"You have a 64-bit processor," \textit{g}\textsubscript{1}: "Your processor runs at 1 GHz or faster," \textit{g}\textsubscript{2}: "Your processor runs at 2 GHz or faster," \textit{r}\textsubscript{1}: "Your processor has at least 1 GB RAM," \textit{r}\textsubscript{2}: "Your processor has at least 2 GB RAM," \textit{h}\textsubscript{16}: "You have at least 16 GB free hard disk space," and \textit{h}\textsubscript{32}: "Your have at least 32 GB free hard disk space."\\
	
	Conclusion C = \textit{b}\textsubscript{32} $\wedge$ \textit{g}\textsubscript{1} $\wedge$ \textit{r}\textsubscript{1} $\wedge$ \textit{h}\textsubscript{16} $\vee$ \textit{b}\textsubscript{64} $\wedge$ \textit{g}\textsubscript{2} $\wedge$ \textit{r}\textsubscript{2} $\wedge$ \textit{h}\textsubscript{32}
	\subsection*{1.3 30:}
	Show that (p $\vee$ q) $\wedge$ ($\neg$p $\vee$ r) $\rightarrow$ (q $\vee$ r) is a tautology.\\
	
	(p $\vee$ q) $\wedge$ ($\neg$p $\vee$ r) $\rightarrow$ (q $\vee$ r)\\
	= (p $\wedge$ r $\vee$ $\neg$p $\wedge$ q $\vee$ q $\wedge$ r) $\rightarrow$ (q $\vee$ r)\\
	= $\neg$(p $\wedge$ r $\vee$ $\neg$p $\wedge$ q $\vee$ q $\wedge$ r) $\vee$ q $\vee$ r\\
	= $\neg$(p $\wedge$ r) $\wedge$ $\neg$($\neg$p $\wedge$ q) $\wedge$ $\neg$(q $\wedge$ r) $\vee$ q $\vee$ r\\
	= ($\neg$p $\vee$ $\neg$r) $\wedge$ (p $\vee$ $\neg$q) $\wedge$ ($\neg$q $\vee$ $\neg$r) $\vee$ q $\vee$ r\\
	= p $\wedge$ $\neg$r $\vee$ $\neg$p $\wedge$ $\neg$q $\vee$ q $\vee$ r\\
	= p $\wedge$ q $\wedge$ $\neg$r $\vee$ p $\wedge$ $\neg$q $\wedge$ $\neg$r $\vee$ $\neg$p $\wedge$ $\neg$q $\wedge$ r $\vee$ $\neg$p $\wedge$ $\neg$q $\wedge$ $\neg$r $\vee$ q $\vee$ r\\
	= q $\vee$ r $\vee$ $\neg$q $\wedge$ $\neg$r\\
	= $\neg$($\neg$q $\wedge$ $\neg$r) $\vee$ $\neg$q $\wedge$ $\neg$r\\
	= \textbf{T}
\end{document}