\documentclass[12pt,en,a4paper]{report}
\usepackage{vntex}
\usepackage{enumerate}
\usepackage{array}
\usepackage{enumitem}
\usepackage{amssymb}
\usepackage{comment}
\usepackage{hyperref}
\usepackage[open, openlevel=1]{bookmark}

\begin{document}
	\begin{table}[]
		\centering
		\begin{tabular}{|p{0.1\textwidth}|p{0.5\textwidth}|p{0.2\textwidth}|p{0.2\textwidth}|}
			\hline
			\multicolumn{4}{|c|}{GROUP --- MEMBER LIST} \\
			\hline
			No. & Name & ID & Role \\
			\hline
			1 & Nguyễn Hoàng & 1952255 & Blank \\
			\hline
			2 & Blank & Blank & Blank \\
			\hline
		\end{tabular}
	\end{table}
\pdfbookmark[0]{Problem 1}{prob1}
	\section*{Problem 1:}
	Let F(x,y) be the statement "x can fool y", where the domain consists of all people in the world. Use quantifiers to express each of these statements.
	\begin{enumerate}
		\item Everybody can fool Fred.\\
		$\forall x F(x, Fred)$
		\item Evelyn can fool everybody.\\
		$\forall x F(Evelyn, x)$
		\item Everybody can fool somebody.\\
		$\forall x \exists y F(x, y)$
		\item There is no one who can fool everybody.\\
		$\neg \exists x \forall y F(x, y)$
		\item Everyone can be fooled by somebody.\\
		$\forall y \exists x F(x, y)$
		\item No one can fool both Fred and Jerry.\\
		$\neg \exists x (F(x, Fred) \wedge F(x, Jerry))$
		\item There is exactly one person whom everybody can fool.\\
		$\exists y (\forall x F(x, y) \wedge (\forall z (\forall w F(w, z))\rightarrow y=z))$
		\item No one can fool himself or herself.\\
		$\neg x F(x, x)$
		\item There is someone who can fool exactly one person besides himself or herself.\\
		$\exists x \exists y (F(x, y) \wedge (\forall z (F(x, z) \rightarrow y = z))$
	\end{enumerate}
\newpage
\pdfbookmark[0]{Problem 2}{prob2}
	\section*{Problem 2:}
	Use rules of inference to show that if $\forall x (P(x) \vee Q(x))$, $\forall x (\neg Q(x) \vee S(x))$, $\forall x (R(x) \rightarrow \neg S(x))$ and $\exists x \neg P(x)$ are true, then $\exists x \neg R(x)$ is true.\\
	\begin{enumerate}
		\item $\forall x (P(x) \vee Q(x))$ (Premise)
		\item $P(c) \vee Q(c)$ (Universal instantiation from (1))
		\item $\forall x (\neg Q(x) \vee S(x))$ (Premise)
		\item $\neg Q(c) \vee S(c)$ (Disjunctive instantiation from (3))
		\item $\forall x (R(x) \rightarrow \neg S(x))$ (Premise)
		\item $R(c) \rightarrow \neg S(c)$ (Universal instantiation from (5))
		\item $\exists x \neg P(x))$ (Premise)
		\item $\neg P(c)$ (Existential instantiation from (7))
		\item $Q(c)$ (Disjunctive syllogism from (2) and (8))
		\item $S(c)$ (Disjunctive syllogism from (4) and (9))
		\item $\neg R(c)$ (Modus tollens from (6) and (10))
		\item $\exists x \neg R(x)$ (Existential generalization from (11))
	\end{enumerate}
\newpage
\pdfbookmark[0]{Problem 3}{prob3}
	\section*{Problem 3:}
	Let L(x, y) be the statement "x loves y", where the domain for both x and y consists of all people in the world. Translate the following predicate logic into English:
	\begin{enumerate}
		\item $\forall L(x, Jerry)\\$
		Everybody loves Jerry.
		\item $\forall x \exists y L(x, y)$\\
		Everybody loves somebody.
		\item $\exists y \forall x L(x, y)$\\
		There is somebody that everybody loves.
		\item $\exists x \neg L(Lydia, x)$\\
		There is somebody that Lydia does not love.
		\item $\exists x \forall y \neg L(y, x)$\\
		There is somebody that no one loves.
		\item $\exists x (\forall y L(y, x) \wedge \forall z (\forall w L(w, z)) \rightarrow z = x)$\\
		There is exactly one person that everybody loves.
	\end{enumerate}
\newpage
\pdfbookmark[0]{Problem 4}{prob4}
	\section*{Problem 4:}
	For each of these arguments, determine whether the argument is correct or incorrect, and explain why.
	\begin{enumerate}
		\item Everyone enrolled in the university has lived in a dormitory. Mia has never lived in a dormitory. Therefore, Mia is not enrolled in the university.\\
		
		The argument is correct. It is an application of modus tollens.
		\item A convertible car is fun to drive. Isaac's car is not a convertible. Therefore, Isaac's car is not fun to drive.\\
		
		The argument is incorrect. It is a case of fallacy of denying the hypothesis.
		\item Quincy likes all action movies. Quincy likes the movie Eight Men Out. Therefore, Eight Men Out is an action movie.\\
		
		The argument is incorrect. It is a case of fallacy of affirming the consequent.
		\item All lobster men set at least a dozen traps. Hamilton is a lobster man. Therefore, Hamilton sets at least a dozen traps.\\
		
		The argument is correct. It is an application of universal instantiation.
	\end{enumerate}
\newpage
\pdfbookmark[0]{Problem 5}{prob5}
	\section*{Problem 5:}
	Express each of these mathematical statements using predicates, quantifiers, logical connective, and mathematical operators, where the domain consists of all integers.
	\begin{enumerate}
		\item The product of two negative integers is positive.\\
		$(x<0 \wedge y<0) \rightarrow (x*y>0)$
		\item The average of two positive integers is positive.\\
		$(x>0 \wedge y>0) \rightarrow \frac{x+y}{2} > 0$
		\item The absolute value of the sum of two integers does not exceed the sum of the absolute values of the integers.\\
		$|x+y| \leq |x|+|y|$
	\end{enumerate}
\newpage
\pdfbookmark[0]{Problem 6}{prob6}
	\section*{Problem 6:}
	Express each of these statements using quantifiers. Then form the negation of the statement so that no negation is to the left of a quantifier. Next, express the negation in simple English.
	\begin{enumerate}
		\item All dogs have fleas.\\
		$\forall x (D(x) \rightarrow F(x))$\\
		$\exists x (D(x) \rightarrow \neg F(x))$\\
		There is a dog that does not have fleas.
		\item There is a horse that can add.\\
		$\exists x (H(x) \wedge A(x))$\\
		$\forall x (H(x) \rightarrow \neg A(x))$\\
		No horse can add.
		\item Every koala can climb.\\
		$\forall x (K(x) \rightarrow C(x))$\\
		$\exists x (K(x) \wedge \neg C(x))$\\
		There is a koala than cannot climb.
		\item No monkey can speak French.\\
		$\forall x (M(x) \wedge \neg F(x))$\\
		$\exists x (M(x) \wedge F(x))$\\
		There is a monkey that can speak French.
		\item There exists a pig that can swim and catch fish.\\
		$\exists x (P(x) \wedge S(x) \wedge C(x))$\\
		$\forall x (P(x) \rightarrow (\neg S(x) \vee \neg C(x)))$\\
		No pigs can swim or catch fish.
	\end{enumerate}
\newpage
\pdfbookmark[0]{Problem 7}{prob7}
	\section*{Problem 7:}
	Use rules of inference to prove that:\\
	\begin{table}[h]
		\centering
		\begin{tabular}{c l}
			$\neg t \vee u$ & (1)\\
			$r \rightarrow (s \vee t)$ & (2)\\
			$(\neg p \vee q) \rightarrow r$ & (3)\\
			$\neg (s \vee u)$ & (4)\\
			\hline
			$\therefore p$ & {}\\
		\end{tabular}
	\end{table}\\
	\begin{tabular}{r l l}
		1. & $\neg t \vee u$ & Premise\\
		2. & $r \rightarrow (s \vee t)$ & Premise\\
		3. & $(\neg p \vee q) \rightarrow r$ & Premise\\
		4. & $\neg (s \vee u)$ & Premise\\
		5. & $\neg s \wedge \neg u$ & De Morgan's law from (4)\\
		6. & $\neg s$ & Simplification from (5)\\
		7. & $\neg u$ & Simplification from (5)\\
		8. & $\neg t$ & Disjunctive syllogism from (1) and (7)\\
		9. & $\neg s \wedge \neg t$ & Conjunction from (6) and (8)\\
		10. & $\neg (s \vee t)$ & De Morgan's Law from (9)\\
		11. & $(\neg p \vee q) \rightarrow (s \vee t)$ & Hypothetical syllogism from (2) and (3)\\
		12. & $\neg (\neg p \vee q)$ & Modus tollens from (10) and (11)\\
		13. & $p \wedge \neg q$ & De Morgan's Law from (12)\\
		14. & $p$ & Simplification from (13)\\
	\end{tabular}
\newpage
\pdfbookmark[0]{Problem 8}{prob8}
	\section*{Problem 8:}
	Use the rules of inference for quantified statements to prove that:\\
	\begin{table}[h]
		\centering
		\begin{tabular}{c l}
			$\forall x \in \mathbb{R}(P(x) \vee Q(x))$ & (1)\\
			$\forall x \in \mathbb{R}(\neg P(x) \wedge (Q(x) \rightarrow T(x)))$ & (2)\\
			\hline
			$\therefore \forall x \in \mathbb{R}(\neg T(x) \rightarrow P(x))$ & {}\\
		\end{tabular}
	\end{table}\\
	\begin{tabular}{r r l}
		1. & $\forall x \in \mathbb{R}(\neg P(x) \wedge (Q(x) \rightarrow T(x)))$ & Premise\\
		2. & $\neg P(c) \wedge (Q(c) \rightarrow T(c)$ & Universal instantiation from (1)\\
		3. & $\neg P(c)$ & Simplification from (2)\\
		4. & $Q(c) \rightarrow T(c)$ & Simplification from (2)\\
		5. & $\forall x \in \mathbb{R} (P(x) \vee Q(x))$ & Premise\\
		6. & $P(c) \vee Q(c)$ & Universal instantiation from (5)\\
		7. & $Q(c)$ & Disjunctive syllogism from (3) and (6)\\
		8. & $T(c)$ & Modus ponens from (4) and (7)\\
		9. & $T(c) \wedge \neg P(c)$ & Addition from (3) and (8)\\
		10. & $\neg T(c) \rightarrow P(c)$ & Logical equivalence from(9)\\
		11. & $\forall x \in R (\neg T(x) \rightarrow P(x))$ & Universal generalization from (10)\\
	\end{tabular}
\newpage
\pdfbookmark[0]{Rosen's Book}{rosen}
	\section*{Rosen's book}
\pdfbookmark[1]{1.4}{1.4}
	\subsection*{1.4:}
\pdfbookmark[2]{9}{1.4.9}
	\subsubsection*{9:}
	Let P(x) be the statement "x can speak Russian" and let Q(x) be the statement "x knows the computer language C++." Express each of these sentences in terms of P(x), Q(x), quantifiers, and logical connectives. The domain for quantifiers consists of all students at your school.
	\begin{enumerate}[label=\textbf{\alph*)}]
		\item There is a student at your school who can speak Russian and who knows C++.\\
		$\exists x (P(x) \wedge Q(x))$
		\item There is a student at your school who can speak Russian but who doesn't know C++.\\
		$\exists x (P(x) \wedge \neg Q(x))$\\
		\item Every student at your school either can speak Russian or knows C++.\\
		$\forall x (P(x) \oplus Q(x))$
		\item No student at your school can speak Russian or knows C++.\\
		$\neg \exists x (P(x) \vee Q(x))$
	\end{enumerate}
	\pdfbookmark[2]{10}{1.4.10}
	\subsubsection*{10:}
	Let C(x) be the statement "x has a cat," let D(x) be the statement "x has a dog," and let F(x) be the statement "x has a ferret." Express each of these statements in terms of C(x), D(x), F(x), quantifiers, and logical connectives. Let the domain consist of all students in your class.
	\begin{enumerate}[label=\textbf{\alph*)}]
		\item A student in your class has a cat, a dog, and a ferret.\\
		$\exists x (C(x) \wedge D(x) \wedge F(x))$
		\item All students in your class have a cat, a dog, or a ferret.\\
		$\forall x (C(x) \vee D(x) \vee F(x))$
		\item Some student in your class has a cat, a ferret, but not a dog.\\
		$\exists x (C(x) \wedge F(x) \wedge \neg D(x))$
		\item No student in your class has a cat, a dog, and a ferret.\\
		$\neg \exists x (C(x) \wedge D(x) \wedge F(x))$
		\item For each of the three animals, cats, dogs, and ferrets there is a student in your class who has this animal as a pet.\\
		$\exists x (C(x) \oplus D(x) \oplus F(x))$
	\end{enumerate}
\pdfbookmark[2]{12}{1.4.12}
	\subsubsection*{12:}
	Let Q(x) be the statement "$x+1>2x$." If the domain consists of all integers, what are these truth values?
	\begin{enumerate}[label=\textbf{\alph*)}]
		\item $Q(0)$\\
		$\Rightarrow$ True
		\item $Q(-1)$\\
		$\Rightarrow$ True
		\item $Q(1)$\\
		$\Rightarrow$ False
		\item $\exists x Q(x)$\\
		$\Rightarrow$ True
		\item $\forall x Q(x)$\\
		$\Rightarrow$ False
		\item $\exists x \neg Q(x)$\\
		$\Rightarrow$ True
		\item $\forall x \neg Q(x)$\\
		$\Rightarrow$ False
	\end{enumerate}
\pdfbookmark[2]{13}{1.4.13}
	\subsubsection*{13:}
	Determine the truth value of each of these statements if the domain consists of all integers.
	\begin{enumerate}[label=\textbf{\alph*)}]
		\item $\forall n(n+1>n)$\\
		$\Rightarrow$ True
		\item $\exists n (2n=3n)$\\
		$\Rightarrow$ True
		\item $\exists n (n=-n)$\\
		$\Rightarrow$ True
		\item $\forall n (3n \leq 4n)$\\
		$\Rightarrow$ True
	\end{enumerate}
\pdfbookmark[2]{18}{1.4.18}
	\subsubsection*{18:}
	Suppose that the domain of the propositional function P(x) consists of the integers -2, -1, 0, 1, and 2. Write out each of these propositions using disjunctions, conjunctions, and negations.
	\begin{enumerate}[label=\textbf{\alph*)}]
		\item $\exists xP(x)$\\
		$P(-2) \vee P(-1) \vee P(0) \vee P(1) \vee P(2)$
		\item $\forall x P(x)$\\
		$P(-2) \wedge P(-1) \wedge P(0) \wedge P(1) \wedge P(2)$
		\item $\exists x \neg P(x)$\\
		$\neg P(-2) \vee \neg P(-1) \vee \neg P(0) \vee \neg P(1) \vee \neg P(2)$
		\item $\forall x \neg P(x)$\\
		$\neg P(-2) \wedge \neg P(-1) \wedge \neg P(0) \wedge \neg P(1) \wedge \neg P(2)$
		\item $\neg \exists xP(x)$\\
		$\neg (P(-2) \vee P(-1) \vee P(0) \vee P(1) \vee P(2))$
		\item $\neg \forall x P(x)$\\
		$\neg(P(-2) \wedge P(-1) \wedge P(0) \wedge P(1) \wedge P(2))$ 
	\end{enumerate}
\pdfbookmark[2]{23}{1.4.23}
	\subsubsection*{23:}
	Translate in two ways each of these statements into logical expressions using predicates, quantifiers, and logical connectives. First, let the domain consist of the students in your class and second, let it consist of all people.
	\begin{enumerate}[label=\textbf{\alph*)}]
		\item Someone in your class can speak Hindi.\\
		$\exists x H(x)$\\
		$\exists y (C(y) \wedge H(y))$
		\item Everyone in your class is friendly.\\
		$\forall x F(x)$\\
		$\forall y (C(x) \rightarrow F(x))$
		\item There is a person in your class who was not born in California.\\
		$\exists x \neg P(x)$\\
		$\exists y (C(x) \wedge \neg P(x))$
		\item A student in your class has been in a movie.\\
		$\exists x M(x)$\\
		$\exists y (C(y) \wedge M(y))$
		\item No student in your class has taken a course in logic programming.\\
		$\forall x \neg L(x)$\\
		$\forall y (C(y) \rightarrow \neg L(y))$
	\end{enumerate}
\pdfbookmark[2]{44}{1.4.44}
	\subsubsection*{44:}
	Determine whether $\forall x (P(x) \leftrightarrow Q(x))$ and $\forall x P(x) \leftrightarrow \forall x Q(x)$ are logically equivalent. Justify your answer.\\\\
	They are not logically equivalent. E.g. let the domain be \{1; 2; 3\} where P(1), Q(1), Q(2) is true, P(2), P(3), Q(3) is false.\\
	Then $\forall x P(x) \leftrightarrow \forall x Q(x)$ is true but $\forall x (P(x) \leftrightarrow Q(x))$ is not true.
\newpage
\pdfbookmark[1]{1.5}{1.5}
	\subsection*{1.5:}
	\pdfbookmark[2]{8}{1.5.8}
	\subsubsection*{8:}
	Let Q(x, y) be the statement "student x has been a contestant on quiz show y." Express each of these sentences in terms of Q(x, y), quantifiers, and logical connectives, where the domain for x consists of all students at your school and for y consists of all quiz shows on television.
	\begin{enumerate}[label=\textbf{\alph*)}]
		\item There is a student at your school who has been a contestant on a television quiz show.\\
		$\exists x \exists y Q(x, y)$
		\item No student at your school has ever been a contestant on a television quiz show.\\
		$\neg \exists x \exists y Q(x, y)$
		\item There is a student at your school who has been a contestant on \textit{Jeopardy} and on  \textit{Wheel of Fortune}.\\
		$\exists x (Q(x, Jeopardy) \wedge Q(x, Wheel of Fortune))$
		\item Every television quiz show has had a student from your school as a contestant.\\
		$\forall y \exists x Q(x, y)$
		\item At least two students from your school have been contestants on \textit{Jeopardy}.\\
		$\exists x \exists y (Q(x, Jeopardy) \wedge Q(y, Jeopardy))$
	\end{enumerate}
\pdfbookmark[2]{10}{1.5.10}
	\subsubsection*{10:}
	Let F(x, y) be the statement "x can fool y," where the domain consists of all people in the world. Use quantifiers to express each of these statements.
	\begin{enumerate}[label=\textbf{\alph*)}]
		\item Everybody can fool Fred.\\
		$\forall x F(x, Fred)$
		\item Evelyn can fool everybody.\\
		$\forall x F(Evelyn, x)$
		\item Everybody can fool somebody.\\
		$\forall x \exists y F(x, y)$
		\item There is no one who can fool everybody.\\
		$\neg \exists x \forall y F(x, y)$
		\item Everyone can be fooled by somebody.\\
		$\forall y \exists x F(x, y)$
		\item No one can fool both Fred and Jerry.\\
		$\neg \exists x (F(x, Fred) \wedge F(x, Jerry))$
		\item There is exactly one person whom everybody can fool.\\
		$\exists y (\forall x F(x, y) \wedge (\forall z (\forall w F(w, z))\rightarrow y=z))$
		\item No one can fool himself or herself.\\
		$\neg x F(x, x)$
		\item There is someone who can fool exactly one person besides himself or herself.\\
		$\exists x \exists y (F(x, y) \wedge (\forall z (F(x, z) \rightarrow y = z))$
	\end{enumerate}
\pdfbookmark[2]{18}{1.5.18}
	\subsubsection*{18:}
	Express each of these system specifications using predicates, quantifiers, and logical connectives, if necessary.
	\begin{enumerate}[label=\textbf{\alph*)}]
		\item At least one console must be accessible during every fault condition.\\
		$\forall y \exists x A(x, y)$
		\item The e-mail address of every user can be retrieved whenever the archive contains at least one message sent by every user on the system.\\
		$\forall x \exists y (A(x, y) \rightarrow R(x))$
		\item For every security breach there is at least one mechanism that can detect that breach if and only if there is a process that has not been compromised.\\
		$\forall y \exists x (D(x, y) \leftrightarrow \exists z \neg C(z))$
		\item There are at least two paths connecting every two distinct endpoints one the network.\\
		$\forall x \forall y \exists a \exists b (a \neq b \rightarrow (P(a, x, y) \wedge P(b, x, y)))$
		\item No one knows the password of every user on the system except for the system administrator, who knows all passwords.\\
		$\forall x (A(x) \rightarrow \forall y K(x, y)) \wedge \neg \exists x(\neg A(x) \wedge \forall yK(x))$
	\end{enumerate}
\pdfbookmark[2]{28}{1.5.28}
	\subsubsection*{28:}
	Determine the truth value of each of these statements if the domain of each variable consists of all real numbers.
	\begin{enumerate}[label=\textbf{\alph*)}]
		\item $\forall x \exists y (x^2 = y)$\\
		$\Rightarrow$ True
		\item $\forall x \exists y (x=y^2)$\\
		$\Rightarrow$ False
		\item $\exists x \forall y (xy=0)$\\
		$\Rightarrow$ True
		\item $\exists x \exists y (x+y \neq y+x)$\\
		$\Rightarrow$ False
		\item $\forall x (x \neq 0 \rightarrow \exists y (xy=1))$\\
		$\Rightarrow$ True
		\item $\exists x \forall y (y \neq 0 \rightarrow xy=1)$\\
		$\Rightarrow$ False
		\item $\forall x \exists y (x+y=1)$\\
		$\Rightarrow$ True
		\item $\exists x \exists y (x+2y=2 \wedge 2x+4y=5)$\\
		$\Rightarrow$ False
		\item $\forall x \exists y (x+y=2 \wedge 2x-y=1)$\\
		$\Rightarrow$ False
		\item $\forall x \forall y \exists z (z=(x+y)/2)$\\
		$\Rightarrow$ True
	\end{enumerate}
\pdfbookmark[2]{33}{1.5.33}
	\subsubsection*{33:}
	Rewrite each of these statements so that negations appear only within predicates (that is, so that no negation is outside a quantifier or an expression involving logical connectives).
	\begin{enumerate}[label=\textbf{\alph*)}]
		\item $\neg \forall x \forall y P(x, y)$\\
		$\exists x \exists y \neg P(x, y)$
		\item $\neg \forall y \exists x P(x, y)$\\
		$\exists y \forall x \neg P(x ,y)$
		\item $\neg \forall y \forall x (P(x, y) \vee Q(x, y))$\\
		$\exists y \exists x (\neg P(x, y) \wedge \neg Q(x, y))$
		\item $\neg (\exists x \exists y \neg P(x, y) \wedge \forall x \forall y Q(x, y))$\\
		$(\forall x \forall y P(x, y)) \vee (\exists x \exists y \neg Q(x, y))$
		\item $\neg \forall x (\exists y \forall z P(x, y, z) \wedge \exists z \forall y P(x, y, z))$\\
		$\exists x (\forall y \exists z \neg P(x, y, z) \vee \forall z \exists y \neg P(x, y, z))$
	\end{enumerate}
\pdfbookmark[2]{36}{1.5.36}
	\subsubsection*{36:}
	Express each of these statements using quantifiers. Then form the negation of the statement so that no negation is to the left of a quantifier. Next, express the negation in simple English. (Do not simply use the phrase "It is not the case that.")
	\begin{enumerate}[label=\textbf{\alph*)}]
		\item No one has lost more than one thousand dollars playing the lottery.\\
		$\neg \exists x L(x)$\\
		$\exists x L(x)$\\
		Someone has lost more than one thousand dollars playing the lottery.
		\item There is a student in this class who has chatted with exactly one other student.\\
		$\exists x \exists y (x \neq y \wedge C(x, y) \wedge \forall z (C(x,z) \rightarrow (x=z \vee y=z)))$\\
		$\forall x \forall y (\neg C(x, y) \vee x=y \vee \exists z(C(x, z) \wedge (z \neq x \wedge z \neq y)))$\\
		All students have chatted with no one or have chatted with more than 2 people.
		\item No student in this class has sent e-mail to exactly two other students in this class.\\
		$\neg \exists x \exists y \exists z (E(x, y) \wedge E(x, z) \wedge x \neq y \wedge x \neq z \wedge y \neq z \wedge \forall w (E(x,w) \rightarrow (w=x \vee w=y \vee w=z)))$\\
		$\exists x \exists y \exists z (E(x, y) \wedge E(x, z) \wedge x \neq y \wedge x \neq z \wedge y \neq z \wedge \forall w (E(x,w) \rightarrow (w=x \vee w=y \vee w=z)))$\\
		There is a student in this class that has sent e-mail to exactly two other students in this class.
		\item Some student has solved every exercise in this book.\\
		$\exists x \forall y S(x, y)$\\
		$\forall x \exists y \neg D(x,y)$\\
		All students have not solved some exercise in this book.
		\item No student has solved at least one exercise in every section of this book.\\
		$\neg \exists x \exists y \forall z S(x, y, z)$\\
		$\exists x \exists y \forall z S(x, y, z)$\\
		Some student has solved at least one exercise in every section of this book.
	\end{enumerate}
\pdfbookmark[2]{49}{1.5.49}
	\subsubsection*{49:}
	\begin{enumerate}[label=\textbf{\alph*)}]
		\item Show that $\forall x P(x) \wedge \exists x Q(x)$ is logically equivalent to $\forall x \exists y (P(x) \wedge Q(y))$, where all quantifiers have the same nonempty domain.\\
		
		$\forall x P(x) \wedge \exists xQ(x)$\\
		= $\forall x P(x) \wedge \exists y Q(y)$\\
		= $\forall x \exists y (P(x) \wedge Q(x))$
		\item Show that $\forall x P(x) \vee \exists Q(x)$ is equivalent to $\forall x \exists y (P(x) \vee Q(y))$, where all quantifiers have the same nonempty domain.\\
		
		$\forall x P(x) \vee \exists x Q(x)$\\
		= $\forall x P(x) \vee \exists y Q(y)$\\
		= $\forall x \exists y (P(x) \vee Q(x))$
		\end{enumerate}
\newpage
\pdfbookmark[1]{1.6}{1.6}
	\subsection*{1.6:}
\pdfbookmark[2]{3}{1.6.3}
	\subsubsection*{3:}
	What rule of inference is used in each of these arguments?
	\begin{enumerate}[label=\textbf{\alph*)}]
		\item Alice is a mathematics major. Therefore, Alice is either a mathematics major or a computer science major.\\
		$\Rightarrow$ Addition.
		\item Jerry is a mathematics major and a computer science major. Therefore, Jerry is a mathematics major.\\
		$\Rightarrow$ Simplification.
		\item If it is rainy, then the pool will be closed. It is rainy. Therefore the pool is closed.\\
		$\Rightarrow$ Modus ponens.
		\item If it snows today, the university will be closed. The university is not closed today. Therefore it did not snow today.\\
		$\Rightarrow$ Modus tollens.
		\item If I go swimming, then I will stay in the sun too long. If I stay in the un too long, then I will sunburn. Therefore, if I go swimming, then I will sunburn.\\
		$\Rightarrow$ Hypothetical syllogism.
	\end{enumerate}
\pdfbookmark[2]{9}{1.6.9}
	\subsubsection*{9:}
	For each of these collections of premises, what relevant conclusion or conclusions can be drawn? Explain the rules of inference used to obtain each conclusion from the premises.
	\begin{enumerate}[label=\textbf{\alph*)}]
		\item "If I take the day off, it either rains or snows." "I took Tuesday off or I took Thursday off." "It was sunny on Tuesday." "It did not snow on Thursday."\\
		
		Assume:\\
		P(x) = "I take the day x off"\\
		Q(x) = "It rains on day x"\\
		R(x) = "It snows on day x"\\
		
		\begin{tabular}{r r l}
			1. & $\forall x (P(x) \rightarrow (Q(x) \vee R(x)))$ & Premise\\
			2. & $P(Tuesday) \vee P(Thursday)$ & Premise\\
			3. & $\neg (Q(Tuesday) \vee R(Tuesday)$ & Premise\\
			4. & $\neg R(Thursday)$ & Premise\\
			5. & $P(Tuesday) \rightarrow (Q(Tuesday) \vee R(Tuesday))$ & Universal instantiation from (1)\\
			6. & $P(Thursday) \rightarrow (Q(Thursday) \vee R(Thursday))$ & Universal instantiation from (1)\\
			7. & $\neg P(Tuesday)$ & Modus ponens from (3) and (5)\\
			8. & $P(Thursday)$ & Disjunctive syllogism from (2) and (7)\\
			9. & $Q(Thursday) \vee R(Thursday)$ & Modus ponens from (6) and (8)\\
			10. & $Q(Thursday)$ & Disjunctive syllogism from (4) and (9)
		\end{tabular}\\
	
		(7): "I did not take Tuesday off"\\
		(8): "I took Tuesday off"\\
		(10): "It rained on Thursday"
		\item "If I eat spicy foods, then I have strange dreams." "I have strange dreams if there is thunder while I sleep." "I did not have strange dreams."\\
		
		Assume:\\
		p: "I eat spicy foods"\\
		q: "I have strange dreams"\\
		r: "There is thunder while I sleep"\\
		
		\begin{tabular}{r r l}
			1. & $p \rightarrow q$ & Premise\\
			2. & $r \rightarrow q$ & Premise\\
			3. & $\neg q$ & Premise\\
			4. & $\neg p$ & Modus tollens from (1) and (3)\\
			5. & $\neg r$ & Modus tollens from (2) and (3)\\
		\end{tabular}\\
		
		(4): "I did not eat spicy foods"\\
		(5): "There was no thunder while I slept"
		\item "I am either clever of lucky." "I am not lucky." "If I am lucky, then I will win the lottery."\\
		
		Assume:\\
		p: "I am clever"\\
		q: "I am lucky"\\
		r: "I will win the lottery"\\
		
		\begin{tabular}{r r l}
			1. & $p \vee q$ & Premise\\
			2. & $\neg q$ & Premise\\
			3. & $q \rightarrow r$ & Premise\\
			4. & $p$ & Disjunctive syllogism from (1) and (2)
		\end{tabular}\\
		
		(p): "I am clever"
		\item "Every computer science major has a personal computer." "Ralph does not have a personal computer." "Ann has a personal computer."\\
		
		Assume:\\
		P(x): "x is a computer science major"\\
		Q(x): "x has a personal computer"\\
		
		\begin{tabular}{r r l}
			1. & $\forall x (P(x) \rightarrow Q(x))$ & Premise\\
			2. & $\neg Q(Ralph)$ & Premise\\
			3. & $Q(Ann)$ & Premise\\
			4. & $P(Ralph) \rightarrow Q(Ralph)$ & Universal instantiation from (1)\\
			5. & $\neg P(Ralph)$ & Modus tollens from (2) and (4)\\
		\end{tabular}\\
	
		(5): "Ralph is not a computer science major"
		\item "What is good for corporations is good for the United States." "What is good for the United States is good for you." "What is good for corporations is for you to buy lots of stuff."\\
		
		Assume:\\
		p: "Good for corporations"\\
		q: "Good for the United States"\\
		r: "Good for you"\\
		s: "You buy lots of stuff:\\
		
		\begin{tabular}{r r l}
			1. & $p \rightarrow q$ & Premise\\
			2. & $q \rightarrow r$ & Premise\\
			3. & $s \rightarrow p$ & Premise\\
			4. & $p \rightarrow r$ & Hypothetical syllogism from (1) and (2)\\
			5. & $s \rightarrow q$ & Hypothetical syllogism from (1) and (3)\\
			6. & $s \rightarrow r$ & Hypothetical syllogism from (2) and (5)\\
		\end{tabular}\\
	
		(4): "What is good for corporations is good for you"\\
		(5): "You buying lots of stuff is good for the United States"\\
		(6): "You buying lots of stuff is good for you"
		\item "All rodents gnaw their food." "Mice are rodents." "Rabbits do not gnaw their food." "Bats are not rodents."\\
		
		Assume:\\
		P(x): "x are rodents"\\
		Q(x): "x gnaw their food"\\
		
		\begin{tabular}{r r l}
			1. & $\forall x (P(x) \rightarrow Q(x))$ & Premise\\
			2. & $P(Mice)$ & Premise\\
			3. & $\neg Q(Rabbits)$ & Premise\\
			4. & $\neg P(Bats)$ & Premise\\
			5. & $P(Mice) \rightarrow Q(Mice)$ & Universal instantiation (1)\\
			6. & $P(Rabbits) \rightarrow Q(Rabbits)$ & Universal instantiation from (1)\\
			7. & $Q(Mice)$ & Modus ponens from (2) and (5)\\
			8. & $\neg P(Rabbits)$ & Modus tollens from (3) and (6)
		\end{tabular}\\
	
	(7): "Mice gnaw their food"\\
	(8): "Rabbits are not rodents"
	\end{enumerate}
\pdfbookmark[2]{16}{1.6.16}
	\subsubsection*{16:}
	For each of these arguments, determine whether the argument is correct or incorrect, and explain why.
	\begin{enumerate}[label=\textbf{\alph*)}]
		\item Everyone enrolled in the university has lived in a dormitory. Mia has never lived in a dormitory. Therefore, Mia is not enrolled in the university.\\
		The argument is correct. It is an application of modus tollens.
		\item A convertible car is fun to drive. Isaac's car is not a convertible. Therefore, Isaac's car is not fun to drive.\\
		The argument is wrong. It is a case of fallacy of denying the hypothesis.
		\item Quincy likes all action movies. Quincy likes the movie Eight Men Out. Therefore, Eight Men Out is an action movie.\\
		The argument is wrong. It is a case of fallacy of affirming the consequent.
		\item All lobster men set at least a dozen traps. Hamilton is a lobster man. Therefore, Hamilton sets at least a dozen traps.\\
		The argument is correct. It is an application of universal instantiation.
	\end{enumerate}
\begin{comment}
\pdfbookmark[2]{19}{1.6.19}
	\subsubsection*{19:}
	Determine whether each of these arguments is valid. If an argument is correct, what rule of inference is being used? If it is not, what logical error occurs?
	\begin{enumerate}[label=\textbf{\alph*)}]
		\item If $n$ is a real number such that $n>1$, then $n^2>1$. Suppose that $n^2>1$. Then $n>1$.\\
		
		$n \in \mathbb{R} \mid p=n>1; q=n^2>1$\\
		$p \rightarrow q \wedge q \equiv p$\\
		$p \rightarrow q \wedge p \equiv q$\\
		The argument is invalid, this is a case of fallacy of affirming the consequent.
		\item If $n$ is a real number with $n>3$, then $n^2>9$. Suppose that $n^2 \leq 9$. Then $n \leq 3$.\\
		
		$n \in \mathbb{R} \mid p=n>3;q=n^2>9$\\
		$p \rightarrow q \wedge \neg q \equiv \neg p$\\
		This is a valid argument (modus tollens)
		\item If $n$ is a real number with $n>2$, then $n^2>4$. Suppose that $n \leq 2$. Then $n^2 \leq 4$.\\
		
		$n \in \mathbb{R} \mid p=n>2;q=n^2>4$\\
		$p \rightarrow q \wedge \neg p \equiv \neg q$\\
		The argument is invalid, this is a case of fallacy of denying hypothesis.
	\end{enumerate}
	\subsubsection*{22:}
\end{comment}
\pdfbookmark[2]{24}{1.6.24}
	\subsubsection*{24:}
	Identify the error or errors in this argument that supposedly shows that if $\forall x (P(x) \vee Q(x))$ is true then $\forall x P(x) \vee \forall Q(x)$ is true.\\
	\begin{tabular}{r r l}
		1. & $\forall x(P(x) \vee Q(x))$ & Premise\\
		2. & $P(c) \vee Q(c)$ & Universal instantiation from (1)\\
		3. & $P(c)$ & Simplification from (2)\\
		4. & $\forall x P(x)$ & Universal generalization from (3)\\
		5. & $Q(c)$ & Simplification from (2)\\
		6. & $\forall x Q(x)$ & Universal generalization from (5)\\
		7. & $\forall x (P(x) \vee \forall x Q(x))$ & Conjunction from (4) and (6)
	\end{tabular}\\

	Errors at step 3 and 5: Simplification requires a conjunction.
\begin{comment}
\pdfbookmark[2]{34}{1.6.34}
	\subsubsection*{34:}
	The Logic Problem, taken from \textit{WFF'N PROOF, The Game of Logic}, has these two assumptions:
	\begin{enumerate}
		\item "Logic is difficult or not many students like logic."
		\item "If mathematics is easy, then logic is not difficult."
	\end{enumerate}
	By translating these assumptions into statements involving propositional  variables and logical connectives, determine whether each of the following are valid conclusions of these assumptions:
	\begin{enumerate}[label=\textbf{\alph*)}]
		\item That mathematics is not easy, if many students like logic.
		\item That not many students like logic, if mathematics is not easy.
		\item That mathematics is not easy or logic is difficult.
		\item That logic is not difficult or mathematics is not easy.
		\item That if not many students like logic, then either mathematics is not easy or logic is not difficult.\\
		
		Assume:\\
		p: "Logic is difficult"\\
		q: "Many students like logic"\\
		r: "Mathematics is easy"\\
	\end{enumerate}
	\begin{tabular}{r r l}
		1. & $p \vee \neg q$ & Premise\\
		2. & $r \rightarrow \neg p$ & Premise\\
		3. & $\neg r \vee \neg p$ & Logical equivalent from (2)\\
		4. & $\neg (r \wedge p)$ & De Morgan's law from (3)\\
		5. & $p \rightarrow \neg r$ & Logical equivalent from (3)\\
		6. & $\neg q \vee \neg r$ & Resolution from (1) and (3)\\
		7. & $q \rightarrow \neg r$ & Logical equivalent from (6)\\
		8. & $r \rightarrow \neg q$ & Logical equivalent from (6)\\
		9. & $(p \vee \neg q) \wedge (\neg r \vee \neg q)$ & Conjunction from (1) and (6)\\
		10. & $\neg q \vee (p \wedge \neg r)$ & De Morgan's law from (9)\\
		11. & $q \rightarrow (p \wedge \neg r)$ & Logical equivalent from (10)\\
	\end{tabular}
	\begin{enumerate}[label=\textbf{\alph*)}]
		\item $q \rightarrow \neg r$\\
		Step (7), valid.
		\item $\neg r \rightarrow \neg q$\\
		Step (8), invalid.
		\item $\neg r \vee p$\\
		Step (3), invalid.
		\item $\neg p \vee \neg r$\\
		Step (3), valid.
		\item $\neg q \rightarrow (\neg r \vee \neg p)$\\
		Step (11), invalid.
	\end{enumerate}
\end{comment}
\end{document}